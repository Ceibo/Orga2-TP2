\subsection{Eficiencia C vs paralelismo simd ASM}
Nuestra hipótesis es que al operar con vectores la implementación en código ASM, que usa instrucciones SIMD, tendrá menor tiempo de ejecución que la hecha en código C. Para averiguar esto evaluaremos los códigos sobre seis tamaños de matrices: 16x16, 32x32, 64x64, 128x128, 256x256 y 512x512. Decidimos usar estos tamaños porque son los que usa la cátedra y nos parecieron que abarcan un buen rango de tamaños. En cada caso repetiremos las ejecuciones cien veces y entonces promediamos los resultados para obtener el desvío estandar. Luego podamos los valores que caigan por arriba de dos veces el desvío estandar calculado, estos valores son llamados \textbf{outliers}. Entonces promediamos los valores no outliers, los que sobrevivieron a la poda, y de acuerdo al porcentaje de estos y su varianza decidimos si asumir al promedio podado como representante de los datos o no. Adicionamente se variarán los niveles de optimización del compilador: $gcc$ en opción $o0$, $o1$ y $o3$.\\

Se usa CPU de 2 GB de RAM y 2 GHz de velocidad para correr los experimentos. Por lo tanto en un segundo el clock del CPU cicla $2*10*e+12$ veces, donde cada ciclo es llamado \textbf{tick}. Los tiempos de ejecución se muestran en microsegundos (1 segundo $== 10*e+6$ microsegundos). Es decir que un microsegundo equivale a $2*10*e+6$ ticks de clock. Luego en caso de que un experimento tire un tiempo de ejecución de $2*10*e+6$ ticks diremos que su tiempo es de un microsegundo.\\

Antes de testear cada función creamos el $solver$ con $solver\_create(size$, 0.05, 0, 0), porque con esos parámetros testea la cátedra, y establecemos densidad y velocidad inicial.
\subsubsection{Función Solver Set Bound}
Fijamos parámetro matriz en $solver\rightarrow v$ y variamos parámetro $b$. Decidimos esto a causa de que el $b$ influye en el resultado de la función: si $b$ es 1 se evalúa una rama en código, si $b$ es 2 se evalúa otra rama y si $b$ no es ni 1 ni 2 se evalúa rama distinta a las anteriores.
El $b$ varía sobre valores: 1, 2, 3 y 10.
\subsubsection{Función Solver Lin Solve}
Decidimos usar las matrices $solver\rightarrow u$, $solver\rightarrow v$ y variar los otros parámetros porque las matrices sólo aportan sumandos, mientras que los otros parámetros multiplican y dividen.
Evaluaremos con los parámetros $a = 1.0 $, $b$ = 1 y $c = 4.0$, llamado $1erOp$, $a = 0.3$, $b$ = 2 y $c = 2.8$, llamado $2daOp$, $a = 100.0$, $b$ = 3, $c = 20.0$ llamado $3raOp$ y $a = -10.0$, $b$ = 10, $c = 0.02$, llamado $4taOp$. Elegimos estos parámetros porque varían en un rango amplio: desde valores decimales ($1erOp$) hasta valores en centenas ($3raOp$).

\subsubsection{Función Solver Project}
Hemos medido la función variando parámetros sobre cuatro matrices diferentes: $1erOp$ que comienza con valor (0.1, 0.2) en posición (0, 0) y luego a medida que avanzamos en posiciones se incrementa en uno el valor en posición anterior y se asigna ese resultado a posición actual, $2daOp$ con mismo proceso pero comenzando en (0.2, -100), $3eraOp$ comenzando en (-10, 0.08) y $4taOp$ comenzando en (1000, 2000). Decidimos esos valores por el amplio rango que abarcan.


