\subsection{Reflexión y problemáticas encontradas}
\par Notamos que el beneficio principal de implementar nuestro código en lenguaje ensamblador en lugar de hacerlo en un lenguaje de alto nivel como C es tener un mayor control sobre las instrucciones de procesador que efectivamente se ejecutan en nuestro programa. Este control cobró una especial importancia en aquellas funciones del programa que eran susceptibles de procesamiento vectorial.\newline

\par Una problemática encontrada fue identificar en qué casos se puede paralelizar el procesamiento y en cuáles no. Encontramos dos casos de uso para los registros vectoriales. El primer caso corresponde a lectura y escritura de datos: si los datos se encuentran contiguos en memoria los registros vectoriales permiten leer o escribir varios datos en un solo acceso a memoria. El segundo caso corresponde a cálculos: si introducimos varios datos en uno o más registros vectoriales, podemos realizar operaciones en simultáneo con todos los datos. Observamos que en algunos algoritmos se dan los dos casos en simultáneo mientras que en otros solo se puede aplicar uno o ninguno. En particular, observamos que en los algoritmos que requieren operaciones secuenciales (es decir, una operación vectorial que requiere el resultado de otra operación vectorial) no siempre es posible realizar los cálculos en paralelo, pero sí la lectura si los datos se encuentran en posiciones contiguas de memoria.

\subsection{Conclusión de la experimentación}

\par Las implementaciones en lenguaje ensamblador tuvieron un menor tiempo de ejecución que las implementaciones en C, con una diferencia que se mantuvo constante a través de los distintos tamaños de entradas.\newline

\par Por otra parte las optimizaciones de compilador, opción $o1$ y $o3$, no demostraron cambios notables en los resultados respecto a compilar con opción $o0$.
