En computación, un procesador vectorial es una unidad central de proceso que implementa un conjunto de instrucciones que operan sobre arreglos de una dimensión de tamaño fijo, llamados vectores, a diferencia de procesadores escalares, cuyas instrucciones operan únicamente sobre datos individuales.
\par Algunos programas informáticos pueden ser implementados en lenguaje ensamblador utilizando exclusivamente instrucciones escalares o utilizando también instrucciones vectoriales. Para estos programas el procesamiento vectorial constituye una optimización dado que se percibe un incremento en la performance (i.e. tiempo neto de procesamiento).
\par Por otro lado, los compiladores modernos de lenguajes de alto nivel, como GNU C Compiler (GCC) incluyen la posibilidad de realización de diversas optimizaciones que se aprecian en el código ensamblado, entre las que se encuentra la utilización de instrucciones de procesamiento vectorial cuando es posible.
\par Para el estudio comparativo utilizamos una simulación de flujo de fluidos basada en las ecuaciones de Navier-Stokes escrita en lenguaje C por el equipo docente de la materia. Hicimos una implementación alternativa de tres funciones involucradas en la simulación en lenguaje ensamblador x86-64 utilizando instrucciones de procesamiento vectorial. El estudio comparativo consiste en el estudio de la performance de las implementaciones en sendos lenguajes.
\par La presente investigación es importante porque a pesar de que la velocidad de procesamiento y la velocidad de acceso a memoria aumentan a lo largo de los años, la curva que describe el incremento de la primera tiene un mayor orden de magnitud que la curva de la segunda\footnote{http://www.cs.columbia.edu/~sedwards/classes/2012/3827-spring/advanced-arch-2011.pdf}. Esto significa que la brecha es cada vez mayor y se traduce en un costo porcentual creciente de los accesos a memoria sobre el costo total de procesamiento de un programa.
\par Asimismo, esta investigación de caracter educativo es importante para programadores que se desempeñan en la academia o en la industria porque motiva a la optimización del código, a la medición de performance de los programas y a la profundización del conocimiento de lo que sucede en el procesador en la ejecución de programas escritos en lenguajes de alto nivel.